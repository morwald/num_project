\documentclass{article}
\usepackage{amsmath}
\usepackage{amsfonts}
\usepackage{amssymb}
\usepackage{listings}
\usepackage{xcolor}
\usepackage{parskip}

\definecolor{codegreen}{rgb}{0,0.6,0}
\definecolor{codegray}{rgb}{0.5,0.5,0.5}
\definecolor{codepurple}{rgb}{0.58,0,0.82}
\definecolor{backcolour}{rgb}{0.95,0.95,0.92}

\lstdefinestyle{mystyle}{
    backgroundcolor=\color{backcolour},   
    commentstyle=\color{codegreen},
    keywordstyle=\color{magenta},
    numberstyle=\tiny\color{codegray},
    stringstyle=\color{codepurple},
    basicstyle=\ttfamily\footnotesize,
    breakatwhitespace=false,         
    breaklines=true,                 
    captionpos=b,                    
    keepspaces=true,                 
    numbers=left,                    
    numbersep=5pt,                  
    showspaces=false,                
    showstringspaces=false,
    showtabs=false,                  
    tabsize=2
}
\lstset{style=mystyle}
\usepackage[utf8]{inputenc}

\title{%
  Project 1 - Flow in a driven cavity and non conforming mesh coupling \\
  \bigskip
  \large Numerics for Fluids, Structures and Electromagnetics}

\author{Mathieu Grondin, Moritz Waldleben}
\date{January 2022}

\begin{document}

\maketitle

\section*{Introduction}
We consider the following Stokes problem on the domain $\Omega \subset
\mathbf{R}^2$ for the velocity $\mathbf{u} : \Omega \rightarrow \mathbb{R}^2$
and the pressure $p:\Omega \rightarrow \mathbb{R}$ :
\begin{align}
    \label{problem}
    -\Delta \mathbf{u} + \nabla p &= \mathbf{f}  \quad\textrm{in } \Omega\nonumber\\ 
    \textrm{div} \mathbf{u} &= 0 \quad \textrm{in } \Omega \\
    \mathbf{u} &= \mathbf{g} \quad \textrm{on } \partial \Omega  \nonumber
\end{align}
where $\mathbf{f} : \Omega \rightarrow \mathbb{R}^2$ and $\mathbf{g} :
\partial\Omega\rightarrow\mathbb{R}^2$ are two given function. In particular
assume that $\mathbf{f} \in [H^{-1}(\Omega)]^2=\mathbf{H}^{-1}(\Omega)$ and
$\mathbf{g} \in [H^{1/2}(\partial\Omega)]^2=\mathbf{H}^{1/2}(\partial\Omega)$.

This problem corresponds to solving a flow of a incompressible fluid where
viscous forces dominate inertial forces (i.e. low Reynolds number). The
equations are essentially the linearized Navier-Stokes equations neglecting
inertial terms. 

\section{Question 1}
Suppose, ab absurdo, that there exists a solution $\mathbf{u}$ to problem \ref{problem} but \\
$\int_{\partial\Omega}\mathbf{g}\cdot\mathbf{n} \neq 0$. It results that : 
\begin{align*}
    \int_{\partial\Omega}\mathbf{g}\cdot\mathbf{n} = \int_{\partial\Omega}\mathbf{u}\cdot \mathbf{n} = \int_{\Omega}\textrm{div}\mathbf{u}=\int_{\Omega}0=0 
\end{align*}
We used the fact that $\mathbf{u}$ is a solution to the equation. Furthermore
the divergence theorem was applied. This results however in a contradiction to
the assumption. Hence this proves that  $\int_{\partial\Omega}
\mathbf{g}\cdot\mathbf{n}=0$ is a necessary condition for the existence of a
solution.

From a physical point of view this correspond to the fact that there can be no
net flux outwards from the domain $\Omega$ as the flow has no source of fluid
inside the domain.

\section*{Question 2}
\subsection*{Weak formulation}
Problem \ref{problem} has to be expressed in a weak formulation. To do so we
multiply the first equation by a test function $\mathbf{v}$ and integrate over
the domain $\Omega$. Hence we get :
\begin{align*}
	& \int_{\Omega}-\Delta \mathbf{u} \,\mathbf{v}+\int_{\Omega}\nabla p \,\mathbf{v} = \int_{\Omega}\mathbf{f}\, \mathbf{v} \\
	\Leftrightarrow& \int_{\Omega} \nabla \mathbf{u} \mathbf{:} \nabla \mathbf{v}-\int_{\partial\Omega} (\nabla \mathbf{u}\cdot n)\, \mathbf{v}  -\int_{\Omega} p\, \mathrm{div}\mathbf{v}+\int_{\partial\Omega}p\,(\mathbf{v}\cdot \mathbf{n})=\int_{\Omega}\mathbf{f}\, \mathbf{v} \\
	\Leftrightarrow& \int_{\Omega} \nabla \mathbf{u} \mathbf{:} \nabla \mathbf{v}-\int_{\Omega} p\, \mathrm{div}\mathbf{v}=\int_{\Omega}\mathbf{f}\, \mathbf{v}+\int_{\partial\Omega} (\nabla \mathbf{g}\cdot \mathbf{n})\, \mathbf{v}-\int_{\partial\Omega}p\,(\mathbf{v}\cdot \mathbf{n})   
\end{align*}
In the development above we did an integration by parts and then used the fact
that $\mathbf{u}=\mathbf{g} \textrm{ on } \partial\Omega$.

Analogously we multiply the second equation by a another test function
$\mathbf(q)$ and integrate over $\Omega$ , using the same properties the
equation becomes :
\begin{equation*}
    -\int_{\Omega} q\, \mathrm{div}\mathbf{u}+\int_{\partial\Omega}q\,(\mathbf{u}\cdot \mathbf{n})=0
\end{equation*}

\subsection*{Functional spaces}
It remains the question which are the functional spaces to be imposed for the
solution for our problem. In the first equation the gradient of the
testfunction $\mathbf{v}$ appears. In addition we wan't make the integral for
the test function vanish on the boundary. A natural choice is :
$[H^1_0(\Omega)]^2 = \mathbf{H}^1_0(\Omega)$
This space is formally defined by :  
\begin{align*}
	\mathbf{H}^1_0(\Omega) = \left\{\mathbf{u} \;:\; \mathbf{u} \in L^2(\Omega),
	\nabla \mathbf{u} \in L^2(\Omega), \textrm{ and }
	\mathbf{u}|_{\partial\Omega}=0 \right\}
\end{align*}

To be changed:
\textbf{Thanks to this definition on the space the two border integrals will vanish.
For the second test functions, nothing special appears, nothing more than
integral over the domain, so a natural space is $L^2(\Omega)$. 
On the other hand, for the spaces of solutions, to apply the Lax-Milgram theorem we need the
same space functions. However in this case this is not possible. First we have
a boundary condition on the border for $\mathbf{u}$. So we consider the affine
space $[H^1_g(\Omega)]^2=\mathbf{H}^1_g(\Omega)$ define as :}
\begin{align*}
    \mathbf{H}^1_g(\Omega)=\left\{\mathbf{u} \;:\; u\in L^2(\Omega), \nabla \mathbf{u} \in L^2(\Omega), \textrm{ and } \mathbf{u}|_{\partial\Omega}=g \right\}
\end{align*}
For the pression, we notice that the solution is unique up to a constant term.
To fix this we can impose that $\int_\Omega p =0$. Appart from this special
issue the pression space is free. We choose $L_0^2(\Omega)$ as the space for
pression, and we impose it also for $q$.
\begin{align*}
    L^2_0(\Omega)=\left\{p\in L^2(\Omega)\; : \; \int_\Omega p =0\right\}
\end{align*}

The problem reads :
Find $(\mathbf{u},p)\in \mathbf{H}^1_g(\Omega)\times L^2_0(\Omega)$ such that
$\forall (\mathbf{v},q)\in \mathbf{H}^1_0(\Omega)\times L^2_0(\Omega)$ the
following holds :
\begin{align}
	\label{weak_problem}
	\int_{\Omega} \nabla \mathbf{u} \mathbf{:} \nabla \mathbf{v}-\int_{\Omega} p\,
	\mathrm{div}\mathbf{v} &= \int_{\Omega}\mathbf{f}\, \mathbf{v} \\
	-\int_{\Omega} q\, \mathrm{div}\mathbf{u} &= 0 \nonumber
\end{align}

\subsection*{Mixed finite element element discrete formulation}
As we have seen in class problem \ref{weak_problem} verifies the
\textbf{inf-sup} condition and admits a unique solution in $\Omega$. No we will
write the inifite dimensional problem in a finite dimensional abstract form :

Find $(\mathbf{u}_h,p)\in \mathbf{V}_h(\Omega) \times Q_h(\Omega)$ such that
$\forall (\mathbf{v}_h,q)\in \mathbf{V}_{h,0}(\Omega) \times Q_{h,0}(\Omega)$ the
following holds :
\begin{align}
	\label{pert_problem}
	a(\mathbf{u}_h,\mathbf{v}_h) - b(\mathbf{v}_h,p_h) &= F(\mathbf{v}) \\
	b(\mathbf{u}_h, q_h) &= 0 \nonumber
\end{align}
The descrete case now has to verify the \textbf{descrete inf-sup} has to be
verified. A common stable choice is the mini-element which consist of an
enreached $\mathbb{P}_1$ finite element space with bubble function and
$\mathbb{P}_1$ FEM space for the pressure.

\subsection*{A priori error estimate}

\section*{Question 3}
We implement the problem with the functional spaces given in the previous
section. The complete code can be seen in annex \ref{annex}. To fix the
constant part of the pressure (we don't have any pressure boundary conditions)
we solve a slightly different problem :

Find $(\mathbf{u}^\epsilon_h,p)\in \mathbf{V}_h(\Omega) \times \mathbf{Q}_h(\Omega)$ such that
$\forall (\mathbf{v}_h,q)\in \mathbf{V}_{h,0}(\Omega) \times \tilde{Q}_h(\Omega)$ the
following holds :
\begin{align}
	\label{abstract_form_dis}
	a(\mathbf{u}^\epsilon_h,\mathbf{v}^\epsilon_h) -
	b(p^\epsilon_h,\mathbf{v}^\epsilon_h) &= F(\mathbf{v}) \\ b(p,\mathbf{v}) -
	c(p^\epsilon_h,q^\epsilon_h) &= 0 \nonumber
\end{align}

This equations are called the perturbed Stokes equations. It is simple to
verify that the zero mean pressure condition is verified. .... show that there
exists a unique solution to this problem and that the error of the perturbed
problem is not to far away from the the actual problem we wanted solve:

As a side remark: This strategy is called the penalty method and helps to.... 

As we seen that problems mostly occurs close to the boundary, we redefine the
mesh to be finer close to the boundary. The space will be divided in two
disjoint regions, specifically we will define an inner square of length 0.8.
Inside this inner square the meshsize will be quite large, but outside this
square the mesh size will be more fine grained. We can  modify the mesh size by
setting the number of points along the inner and outer boundary. 

\begin{lstlisting}[language=C++]
\end{lstlisting}


This definition of the mesh gives us :

With this mesh the solution is given by :

\section*{Question 4}

\section*{Question 5}
J'ai compris.

\section*{Annex} \label{annex}
\begin{lstlisting} [language=C++]
\end{lstlisting}

\end{document}
